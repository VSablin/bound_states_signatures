\documentclass[aps,pra,twocolumn,floatfix,superscriptaddress]{revtex4-1}%,twocolumn,draft
\usepackage{amsmath}
\usepackage{amssymb}
\usepackage{xcolor}
\usepackage{graphicx}
\usepackage{dcolumn}
\usepackage{braket}

\def\boldrm#1{{\bf #1}}
\def\boldit#1{\mbox{\boldmath$#1$}}
\def\calbf#1{\mbox{\boldmath${\cal #1}$}}
\def\q{\quad}
\def\k{\mbox{\boldmath$\kappa$}}
\newcommand{\eq}[1]{Eq.~\eqref{#1}}

\begin{document}

\title{Dynamical signatures of bound states in waveguide QED}

\author{E. S\'anchez-Burillo}
\affiliation{Instituto de Ciencia de Materiales de Aragon and Departamento de Fisica de la Materia Condensada, CSIC-Universidad de Zaragoza, E-50012 Zaragoza, Spain}
\author{D. Zueco}
\affiliation{Instituto de Ciencia de Materiales de Aragon and Departamento de Fisica de la Materia Condensada, CSIC-Universidad de Zaragoza, E-50012 Zaragoza, Spain}
\affiliation{Fundacion ARAID, Paseo Maria Agustin 36, E-50004 Zaragoza, Spain}
\author{L. Mart\'in-Moreno}
\affiliation{Instituto de Ciencia de Materiales de Aragon and Departamento de Fisica de la Materia Condensada, CSIC-Universidad de Zaragoza, E-50012 Zaragoza, Spain}
\author{J. J. Garc\'ia-Ripoll}
\affiliation{Instituto de Fisica Fundamental, IFF-CSIC, Calle Serrano 113b, Madrid E-28006}
\begin{abstract}
In this work we study the spontaneous decay of an impurity coupled to a linear array of bosonic cavities forming a single band waveguide. The frequency of the emitted photon is different from the resonance single-photon scattering frequency, which perfectly matches the bare frequency of the impurity. This breaks down the typical analogy between spontaneous emission and scattering. We study how the position of the impurity energy with respect to the photonic band influences the profile in position space of the emitted photon. In addition, the impurity follows a rich dynamics: in the beginning it shows an exponential decay, followed by  a power-law tail and finally it achieves an oscillatory stationary regime. The parameters needed to see these effects are within the state-of-the-art of several quantum technologies.
\end{abstract}

\maketitle

\section{Introduction}



Interaction between few level systems or quantum impurities and photonic media with nonlinear dispersion relations and band gaps implies a plethora of interesting phenomena \cite{Lambropoulos2000}, e.g., modification of the level structure of the impurity \cite{John1990,John1991,Ripoll2015}, non-trivial dynamics \cite{John1994,Gaveau1995,Redchenko2014} or charge transfer enhancement \cite{Tanaka2006}. %
Perhaps one of the most striking phenomena in photonic systems with bandgaps and quantum impurities is the appearance of bound states\ \cite{John1984,John1987} where a photonic excitation is confined to the vicinity of the impurity. This idea has been studied in various theoretical works, with a particular focus on multi-photon bound states\ \cite{Rabl2015,Cirac2015}. However, recent experiments have shown that already at the single-photon level it is possible to engineer and spectroscopycally probe bound states, using cavity arrays and transmon qubits for the photonic medium and the quantum impurity, respectively\ \cite{liu2016}.


The purpose of this paper is to study the decay of a quantum impurity embedded in a photonic reservoir with a certain band gap. The medium is a one-dimensional array of coupled bosonic cavities with tight-binding interaction, so we are considering decay in a waveguide-QED scenario. This model gives a cosine-shaped dispersion relation. The influence of the bound states in this system has already been considered in two-photon scattering \cite{Longo2010,Longo2011} or even in the decay of the impurity, but just when its energy is in the middle of the band \cite{Lombardo2014}. We show that the energy of the emitted photon has an energy shift with respect to the energy of the impurity. We see that the impurity decays with some power-law after the initial exponential decay. Both phenomena show signatures of the bound states and the density of states. Lastly, we will study the profile in position space of the emitted field.  
Our calculations are certainly important for  characterizing single photon emitters embedded in photonic mediums.
All the phenomenology presented here, can be observed 
 within the state-of-the-art of different quantum technologies, such as photonic crystals \cite{Arcari2014,Sollner2015,Lodahl2015}, superconducting circuit architectures \cite{liu2016,Astafiev2010,Hoi2011,Hoi2013,VanLoo2013,Hoi2013b}, or cold atoms\ \cite{goban2015,thompson2013}. 



The manuscript is organized as follows. First, we introduce the Hamiltonian and its spectrum in the single-excitation subspace.  In section \ref{sec:scatt} we briefly review the single-photon scattering. In section \ref{sec:spontaneous_decay}, we discuss the main results of the paper. First, we show that an excited impurity emits a photon with a frequency shift depending both on the coupling constant and the energy of the impurity with respect to the photonic band.  Then, we study the field distribution of the emitted photon in position space. Finally,  we characterize the spontaneous emission, which shows an exponential decay followed by a a power-law tail and finally it achieves an oscillatory stationary regime.
We end up with the conclusions. Some technical details are sent to the appendices.


%%%%%%%%%%%%%%%%%%
%%%%%%%%%%%%%%%%%%
%%%%% Model
%%%%%%%%%%%%%%%%%%
%%%%%%%%%%%%%%%%%%
\section{Model}

The photonic medium is represented as a chain of $L$ discrete bosonic sites (we take $L\to\infty$ althrough the paper) coupled to some impurity living at site $x_0=0$. 
The Hamiltonian of the combined system is ($\hbar =1$):
\begin{align}
\label{eq:H} H   = \; &  
\Delta b^\dagger b  + 
\sum_{x=-\infty}^\infty \left(\epsilon a^\dagger_x a_x -  J ( a_{x+1}^\dagger a_x  + a_{x}^\dagger a_{x+1})\right)
 \nonumber\\
&  + g ( b^\dagger \,  a_0 +   a_0^\dagger \, b) \, ,
\end{align}
where $a_{x}$ and $a_{x}^\dagger$  annihilate and create, respectively, a photon at position $x$ and $b$ and $b^\dagger$ annihilate and create excitations at the impurity,
 which can be a two-level system another, resonator, etc. 
$\Delta$ is the energy of the impurity.  
The band of free photons is defined by a dispersion relation depending on both the on-site photon energy $\epsilon$ and the hopping parameter $J$:  $\omega_k = \epsilon  - 2 J \cos k$, being $k$ the momentum and $\omega_k$ the corresponding energy. The momentum $k$ lies into $[-\pi/d,\pi/d)$, with $d$ the lattice spacing. We take $d=1$. The group velocity is $v_k:=d\omega_k/dk=2J\sin k$. 
The interaction Hamiltonian between the impurity and the cavities, the last term in Eq. \eqref{eq:H}, is given by the celebrated Jaynes-Cummings model, being $g$ the coupling constant.
Importantly, our Hamiltonian commutes with the number operator $\mathcal{N}:= \sum_x a_x^\dagger a_x + b^\dagger b$, $[H,\mathcal{N}]=0$, so the number of particles is a conserved quantity. 
A scheme of the system as well as the dispersion relation $\omega_k$ are shown in Fig. \ref{fig:scheme}.


\begin{figure}[thb!]
\includegraphics[width=1\columnwidth]{fig1_imp_gimp.pdf}
\caption{{\bf Scheme of the system and dispersion relation.} (a) Scheme of the system. In the figure we sketch the spontaneous emission. The dispersion relation can be seen in the top right part of the panel (b).}\label{fig:scheme}
\end{figure}

%%%%%%%%%%%%%%%%%%%%%%%%%%%%%%%%%%%%%%%%%%
%%%%%%%%%%%%%%%%%%%%%%%%%%%%%%%%%%%%%%%%%%

\subsection{Single-particle eigenstates}

This model is analytically solvable in the single-excitation subspace. A complete basis is given by the scattering eigenstates $|\Psi_k\rangle$ \cite{Nori2008a} and the bound states $|\Psi_\pm\rangle$ \cite{Longo2010,Longo2011}. The former can be written as,
\begin{align}
\label{eq:scattering_states} 
|\Psi_k\rangle = & \Big [ \sum_{x<0}(e^{ikx}+r_k e^{-ikx})a_x^\dagger 
 +  \sum_{x\geq 0} t_k e^{ikx} a_x^\dagger 
%\\ 
%& 
+ d_k b^\dagger \Big]  |0\rangle ,
\end{align}
being $|0\rangle$ the vacuum state of the system ($a_x|0\rangle=b|0\rangle = 0$).
Te coefficients 
 $t_k$ and $r_k$ play the role of  
 transmission and reflection amplitudes respectively
whenever a plane wave is sent through the impurity [First term in Eq. \eqref{eq:scattering_states}].  They are given by, 
\begin{align}
\label{eq:transmission}
t_k & =\frac{iv_k(\omega_k - \Delta)}{iv_k(\omega_k-\Delta)-g^2} \, , 
\\
r_k&=t_k-1 
\\ 
d_k  &= \frac{g t_k}{\omega_k-\Delta} \,
\label{eq:d_scattering_states}.
\end{align} 
The bound states read
\begin{equation}
 |\Psi_\pm\rangle =  N_\pm \left(\sum_x e^{-\kappa_\pm |x|} a_x^\dagger + d_\pm b^\dagger\right)|0\rangle.\label{eq:bound_states}
\end{equation} 
$N_\pm$ is a normalization constant, $1/\kappa_\pm$ is the localization length and $d_\pm$ is  the impurity amplitude. The energy of $|\Psi_\pm\rangle$ is $\omega_\pm = \epsilon - J(e^{-\kappa_\pm} + e^{\kappa_\pm})$. The expressions of $d_\pm$ and $N_\pm$ (Eqs. \eqref{eq:d_bound_states} and \eqref{eq:Npm} respectively), as well as the computation of $\kappa_\pm$, are shown in the appendix \ref{app:eigen}. Notice that the localization lengths,  $\kappa_\pm$, 
fix the properties for the bound states: the energies $\omega_\pm$, the impurity amplitudes $d_\pm$ and the normalization factors $N_\pm$.


In Fig. \ref{fig:E_bound}  we plot the bound states energy $\omega_\pm$ and the band limits for the propagating (scattering) eigenstates as a function of the coupling constant $g$.
Two cases are shown.  The   impurity energy in the middle of the band ($\Delta-\epsilon=0$, solid lines), and below  ($\Delta-\epsilon=-J$, dotted-dashed lines). 
The bound states are localized (not propagating), thus they are always outside the band limits.
As $g\to 0$, $\omega_{-(+)}$   approach the bottom (top) of the band.
If  the impurity splitting  is in the middle of the band, the energies of the bound states are  symmetric.
On the other hand, $\omega_-$ moves away from the band faster than $\omega_+$ if $\Delta-\epsilon<0$, and vice versa (not shown). Therefore, the position of the impurity energy with respect to the band breaks the symmetry between $\omega_+$ and $\omega_-$.

\begin{figure}[thb!]
\includegraphics[width=1\columnwidth]{E_bound_all.pdf}
\caption{{\bf Bound states.} $(\omega_+-\epsilon)/J$ (solid, red line for $(\Delta-\epsilon)/J=0$ and dotted-dashed, red line for $\Delta-\epsilon=-J$) and $(\omega_--\epsilon)/J$ (solid, blue line for $\Delta-\epsilon=0$ and dotted-dashed, blue line for $\Delta-\epsilon=-J$) as a function of $g/J$. The red and blue dashed lines render the bottom and top band egdes respectively.}\label{fig:E_bound}
\end{figure}
 

%%%%%%%%%%%%%%%%%%%%%%%%%%%%%%%%%%%%%%%%%%
%%%%%%%%%%%%%%%%%%%%%%%%%%%%%%%%%%%%%%%%%%

\section{Scattering dynamics}\label{sec:scatt}
Let us review the single particle scattering dynamics.  
This is a problem deeply studied in the literature.  In the photonic context relevant references are \cite{Nori2008a, Fan2005a, Fan2005b}.
Eq. \eqref{eq:transmission} reveals that, 
 when $\omega_k=\Delta$ there is full reflection ($T_k:=|t_k|^2=0$, $R_k:=|r_k|^2 = 1$). 
 It is shown in Fig. \ref{fig:R}, where $R_k$ is illustrated with respect to $(\omega_k-\epsilon)/J$ and $(\Delta-\epsilon)/J$ for coupling $g=J/2$. 
Considering the  input as a  single photon spectroscopy probe,  we could be tempted to argue that, like in the scattering, the impurity absorption is also maximum at resonance.  We will show that, due to the presence of bound states, this is not the case. 

\begin{figure}[thb!]
\begin{center}
\includegraphics[width=1.\columnwidth]{R_vs_w_Delta_g_0_5.pdf}
\caption{{\bf Reflection probability.} $R_k$ as a function of $(\omega_k-\epsilon)/J$ and $(\Delta-\epsilon)/J$ for $g=J/2$. The black line is $\Delta=\omega_k$. $R_k=1$ as $\omega_k=\Delta$; in addition, it goes to 1 at the band edges, as $\sin k=0$ (Eq. \eqref{eq:transmission}).}\label{fig:R}
\end{center}
\end{figure}

%%%%%%%%%%%%%%%%%%%%%%%%%%%%%%%%%%%%%%%%%%
%%%%%%%%%%%%%%%%%%%%%%%%%%%%%%%%%%%%%%%%%%

\section{Spontaneous decay}\label{sec:spontaneous_decay}

We discuss now   the spontaneous emission and the role of the bound states.
In our studies, the  initial condition is $|\Psi(0)\rangle = b^\dagger |0\rangle$. Spanning this state in  scattering and bound eigenstates, we can compute the state at time $t$:
\begin{align}
|\Psi(t)\rangle = & \frac{1}{2\pi}\int_{-\pi}^\pi dk\;c_ke^{-i\omega_k t}|\Psi_k\rangle \nonumber \\
 + & c_+ e^{-i\omega_+ t} |\Psi_+\rangle + c_- e^{-i\omega_- t} |\Psi_-\rangle, \label{eq:psi(t)}
\end{align}
with $c_k=\langle\Psi_k|b^\dagger |0\rangle$ and $c_\pm=\langle\Psi_\pm|b^\dagger |0\rangle$. These coefficients are obtained by projecting the eigenstates Eq. \eqref{eq:scattering_states} and Eq. \eqref{eq:bound_states}. The expressions are
\begin{align} \label{eq:ck}
c_k &  = \frac{iv_k g}{iv_k(\omega_k - \Delta) + g^2},\\
\label{eq:cpm}
c_\pm & =  \left(\frac{1+e^{-2\kappa_\pm}}{1-e^{-2\kappa_\pm}}+\frac{g^2}{(\omega_\pm - \Delta)^2}\right)^{-\frac{1}{2}} \frac{g}{\omega_\pm - \Delta}.
\end{align}

%%%%%%%%%%%%%%%%%%%%%%%%%%%
%%%%%%%%%%%%%%%%%%%%%%%%%%%
\subsection{Energy shift}

The mean energy of $|\Psi(t)\rangle$ is $\langle H\rangle (t) = \langle H\rangle (0) = \Delta$.  Thus, 
\begin{equation}\label{eq:H(t)}
\Delta  = \frac{1}{2\pi}\int_{-\pi}^\pi dk |c_k|^2 \omega_k + |c_+|^2 \omega_+ + |c_-|^2 \omega_- \, ,
\end{equation}
and the  average energy for the propagating field  is
\begin{equation}
\omega_\text{ph} = \frac{\int_{-\pi}^\pi dk |c_k|^2 \omega_k/2\pi}{\int_{-\pi}^\pi dk |c_k|^2/2\pi} = \frac{\int_{-\pi}^\pi dk |c_k|^2 \omega_k/2\pi}{(1-P_\text{lig})},
\end{equation}
with $P_\text{lig} = |c_+|^2 + |c_-|^2$.  
Using Eq. \eqref{eq:H(t)}  we conclude that
\begin{equation}
\omega_\text{ph} =\frac{\Delta - |c_+|^2 \omega_+ - |c_-|^2 \omega_-}{1-P_\text{lig}}. \label{eq:omega_ph}
\end{equation}
We plot $(\omega_\text{ph}-\epsilon)/J$ as a function of the $(\Delta-\epsilon)/J$, for different values of $g$ in Fig. \ref{fig:E_ph}. The closer $\Delta$  is  the band edges, more different is  $\omega_\text{ph}$ from $\Delta$ (solid straight line).
The shift is larger increasing the coupling $g$
Eventually, as $g/J\to\infty$, the emitted energy is always in the middle of the band.
Therefore, the impurity emits a photon with an energy different than $\Delta$ even though the scattering resonance frequency happens as $\omega_k=\Delta$, unless $\Delta$ is in the middle of the band.
This  breaks the correspondence between scattering and emission spectra. 


\begin{figure}[thb!]
\includegraphics[width=1.0\columnwidth]{E_ph_vs_delta.pdf}
\caption{{\bf Mean photon energy.} Dependence of $(\omega_\text{ph}-\epsilon)/J$ with $(\Delta-\epsilon)/J$ for $g=J/2,J,3J/2,2J$. The straight line renders the diagonal $\omega_\text{ph}=\Delta$.% In the inset we show $\delta\omega_\text{ph}$ vs $g$ in log-log scale. The points are the numerical calculations and the solid lines fits to fourth order polynomials. The color code is the same as in the main plot ($\Delta=-1.5,-1.0,-0.5$ in blue, red and dark green respesctively, now from top to bottom)
}\label{fig:E_ph}
\end{figure}

In order to shed more light on this, we study the energy distribution of the emitted photon $|c_k|^2$. We plot it as a function of $(\omega_k-\epsilon)/J$ and $(\Delta-\epsilon)/J$ in Fig. \ref{fig:c_k} for $g=J/5$ and $g=J/2$ [a) and b) respectively]. If the coupling is small enough (left panel), the energy distribution is well peaked around $\omega_k=\Delta$. However, as $g$ increases (right panel), $|c_k|^2$ reaches its maximum for $\omega_k\neq\Delta$, being the difference bigger as $\Delta$ is closer to one of the band edges. This deviation from $\Delta$ implies the frequency shift of the emitted photon, as already seen in Eq. \eqref{eq:omega_ph} and Fig. \ref{fig:E_ph}.
Therefore,   scattering and emission are not inverse problems but
emission and absorption  are:    $c_k=d_k^*$, in Eqs.  \eqref{eq:d_scattering_states} and \eqref{eq:ck}.

\begin{figure}[thb!]
\includegraphics[width=1.0\columnwidth]{e_vs_w_Delta_g_0_2_0_5.pdf}
\caption{{\bf Emission energy distribution.} $|c_k|^2$ as a function $\omega_k/J$ and $\Delta/J$ for $\epsilon=0$ and a) $g=J/5$, b) $g=J/2$. The black line renders $\Delta=\omega_k$. We normalize $c_k$ such that $\text{max}_k(|c_k|^2)=1$.}\label{fig:c_k}
\end{figure}

In situations where the band limits are sent to infinity, there is not bound states and the impurity emits single photons peaked at its bare frequency, $\Delta$.  In such a case, emission and scattering resonance occurs at the same energy, $\Delta$.   Whenever the bands are considered, as in our case,  the general  condition for vanishing frequency shift is $|c_+|^2(\omega_+-\Delta)=|c_-|^2(\Delta-\omega_-)$, Eq. \eqref{eq:omega_ph}. This condition is fulfilled if $\Delta$ is in the middle of the band, since $\omega_\pm$ move away from the band at the same rate (solid lines of Fig. \ref{fig:E_bound}) and the overlap with both eigenstates $|c_\pm|^2$ is the same, Cf. Eq. \eqref{eq:cpm}. 

Finally, we characterize the emission probability in propagating modes, $P_\text{emission}:=1-P_\text{lig}$, see Fig. \ref{fig:P_emi}.  Two main effects can be extracted.  First, the emission into bound states is negligible ( $P_\text{emission} \cong 1$) in the range $g/J \ll1$.   Increasing this ratio, and depending on the location of the impurity with respect to the band, bound states are populated, decreasing the emission in travelling photons. The latter occurs at smaller values of the coupling as the impurity is closer to the band edges.  Despite this behavior, the emission probability is always appreciable for large values of the ratio: $g/J \cong 2.5$ yields $P_\text{emission} \cong 0.25$ for the values of $\Delta$ considered in Fig. \ref{fig:P_emi}.

\begin{figure}[thb!]
\includegraphics[width=1.0\columnwidth]{p_emission_a.pdf}
\caption{{\bf Probability of emitting.} $P_\text{emission}$ for $\Delta-\epsilon=-3J/2,-J,-J/2,0$ as a function of $g/J$ from bottom to top (solid blue, dashed red, dotted green and dot-dashed black, respectively).}\label{fig:P_emi}
\end{figure}

%%%%%%%%%%%%%%%%%%%%%%%%%%%%%%%%%%%%%%%%
%%%%%%%%%%%%%%%%%%%%%%%%%%%%%%%%%%%%%%%%

\subsection{Emitted field}

We now study the shape of the emitted field. We compute the amplitudes in position space, $\phi_x:=\langle 0|a_x|\Psi(t)\rangle$ [Cf. App.\ \ref{app:field}] to reconstruct the spatial profile of the emitted photon. The probability $|\phi_x|^2$ is shown Fig. \ref{fig:w_n} at time $t=75/J$ for two values of the detuning, $\Delta-\epsilon=0$ (blue solid) and $-J$ (red dashed), and fixed $g=J/5$. The vertical solid black lines represents the causal cone $|x_\text{max}|=v_\text{max}t$, defined in terms of the maximum group velocity $v_\text{max}=v_{k=\pi/2}=2J$. Notice that the support of the photon wavefunction $|\phi_x|^2$ is defined by this causal constraint and coincides for both impurity energies. However, the blue curve corresponding to the impurity being in the middle of the band, is quite sharp around $|x_\text{max}|$, whereas the other ($\Delta-\epsilon=-J$) is peaked around $x$ clearly far away from $|x_\text{max}|$. This is due to the fact that, in momentum space, the first one is distributed around $k=\pi/2$, corresponding to the highest group velocity, whereas the other is peaked around $k$ such that $\omega_k-\epsilon\simeq -J$, with $v_k$ smaller than $v_\text{max}$. This behavior would be completely different in a waveguide with linear dispersion relation, where the emitted photon would be well peaked around $|x_\text{max}|$ in position space, no matters $\Delta$ \cite{Eberly2002}. Notice that the emitted field has a little exponential tail beyond $|x_\text{max}|$, 
wich are nothing but the exponential tails outside the light-cone in the free field scalar propagator \cite[Sect. 4.5]{Greiner-fq}.

\begin{figure}[thb!]
\includegraphics[width=1.0\columnwidth]{wx_g_0_2_Delta_0_-1.pdf}
\caption{{\bf Field distribution.} $|\phi_x|^2$ as a function of $x$ at time $t=75/J$ for $\Delta-\epsilon=0$ (solid blue) and $\Delta-\epsilon=-J$ (dashed red), fixing $g=J/5$. The black solid vertical lines render the propagation limit $|x|=v_\text{max}t$, with $v_\text{max}=v_{k=\pi/2}=2J$.}\label{fig:w_n}
\end{figure}

%%%%%%%%%%%%%%%%%%%%%%%%%%%%%%%%%%%%%%%%
%%%%%%%%%%%%%%%%%%%%%%%%%%%%%%%%%%%%%%%%


\subsection{Impurity dynamics}

Finally, we discuss impurity dynamics. 
Using Eq. \eqref{eq:psi(t)}
\begin{align}
\label{eq:qubit_amplitude}
c_1(t) & =
\langle 0|b|\Psi(t)\rangle
\\ \nonumber
& =
\underset {c_1^\text{s}(t):=}{
\frac{1}{2\pi}\int_{-\pi}^\pi dk|c_k|^2 e^{-i\omega_k t}}  +
\underset {c_1^\text{b}(t):=}{
\sum_{\alpha = \pm}
 |c_\alpha|^2 e^{-i\omega_\alpha t}} \, .
\end{align}
where he have explicitly split  the contribution in the decay  due to coupling to propagating modes,  $c_1^\text{s}(t):=\int_{-\pi}^\pi dk |c_k|^2 e^{-i\omega_k t}/2\pi$, and  to  bound states, $c_1^\text{b}(t):=|c_+|^2 e^{-i\omega_+ t} + |c_-|^2 e^{-i\omega_- t}$.

First, we focus on  $c_1^\text{s}(t)$:
\begin{align}
&c_1^\text{s}(t) =  e^{-i\epsilon t}\frac{4g^2}{\pi J^2}\int_{-1}^1 dy F(y)e^{i2yJt}\label{eq:c_sc_app},\\
&F(y)=\frac{\sqrt{1-y^2}}{4(1-y^2)\left((\Delta-\epsilon)/J+2y\right)^2+(g/J)^4}.\label{eq:F}
\end{align}
\begin{figure}[thb!]
\includegraphics[width=1.0\columnwidth]{integrand_inset_Delta_0_g_0_2.pdf}
\caption{{\bf Integrand.} Integrand $F(y)$ for $g=J/5$ (red, solid), Lorentzian approximation (blue, dashed), $F(y)$ for $g=0$ (black, dotted) and $G(y)$ (Eq. \eqref{eq:G}, black, dotted-dashed). In the inset, we zoom $F(y)$ around $y=-1$ to see $\Delta y_-$. We fix $\Delta=\epsilon$.}\label{fig:integrand}
\end{figure}
The behavior of $c_\text{sc}(t)$ is determined by $F(y)$. We plot it in Fig. \ref{fig:integrand}. If  $\Delta$ is far enough from the band edges and $g$ is small enough, $F(y)$ can be approximated by a Lorentzian $L(y)$ (blue, dashed curve in Fig. \ref{fig:integrand}), giving an exponential decay $c_1^\text{s}(t)\sim e^{-(i\varphi+\gamma_0/2)t}$. 
Besides, 
the first order approximation of $\gamma_0$ and $\varphi$ gives the Fermi golden rule, $\gamma_0=J\sin k_\Delta/g^2$, with $k_\Delta$ such that $\omega_{k_\Delta}=\Delta$, and $\varphi=\Delta$. More details can be found in the appendix \ref{app:integrand}. We show $\gamma_0$ and $\delta\varphi=\varphi-\Delta$ for $g=3J/10$ in Fig. \ref{fig:qubit_decay},   obtained with the Fermi golden rule and the  Lorentzian compared   with the (numerical) exact result.  The Fermi golden rule perfectly matches the results around the middle of the band, but the Lorentzian approximation is necessary as we get closer to the band edges.

\begin{figure}[thb!]
\includegraphics[width=1.0\columnwidth]{gamma_phi_g_0_3.pdf}
\caption{{\bf Exponential decay.} a) $\gamma_0/J$ and b) $\delta\varphi/J$ depending on the position of the impurity energy with respect to the band for $g=3J/10$ and $\epsilon=0$. $\gamma_0/J$ is depicted in units of the Fermi golden rule result in the middle of the band. The blue points are computed numerically, the red, solid curve corresponds to the Fermi golden rule and the dashed, black curve to the single pole approximation. As seen, the single-pole approximation works better close to the band egdes, whereas in the middle of the band both models converge.}\label{fig:qubit_decay}
\end{figure}


If $t\gg\tau_0:=\gamma_0^{-1}$, relaxing dynamics in nontrivial electromagnetic media usually follows a power-law \cite{Gaveau1995}. It is originated by singularities in $F(y)$. We find two different contributions. First, $F(y)$ quasi-diverges with $(1-y^2)^{-1/2}$ as $y\to\pm 1$. In fact, it diverges if $g=0$ (black, dotted curve in Fig. \ref{fig:integrand}). As $g\neq 0$,  shows a narrow peak that  goes to 0 as $y=\pm 1$, becoming maximal for some $y_\pm^*$ close to $\pm 1$ respectively. This kind of singularity with $(1-y^2)^{-1/2}$ gives a $t^{-1/2}$ decay in the scattering amplitude $c_1^\text{s}(t)$ (Appendix \ref{app:integrand}), or a $t^{-1}$ decay in the scattering contribution to the impurity population, $P_1^\text{s}(t):=|c_1^\text{s}(t)|^2$. Not being a real divergence, $P_1^\text{s}(t)$ eventually decays exponentially with $e^{-t/\tau_{1,+}}$ and $e^{-t/\tau_{1,-}}$, where $\tau_{1,\pm}:=(4 \Delta y_\pm)^{-1}$, with $\Delta y_\pm:= |y_\pm \mp 1|$ (see inset of Fig. \ref{fig:integrand}). Finally, as $t\gg \tau_0,\tau_{1,\pm}$, any soft part of $F(y)$ is smeared out because of the extremely fast oscillating factor $e^{i2Jt}$. The only surviving contribution is due to the real singularities of $F(y)$, which happen for $y= \pm 1$, where $F(y)\sim \sqrt{1-y^2}$. $P_1^\text{s}(t)$ decays with $t^{-3}$ and it oscillates with $\cos^2(2t-3\pi/4)$. More details can be found in the Appendix \ref{app:integrand}.


The contribution of the bound states is much simpler: it gives an oscillatory term which persists for infinitely long times, $P_1^\text{b}(t):=|c_1^\text{b}(t)|^2=|c_+|^4+|c_-|^4+2|c_+c_-|^2\cos((\omega_+-\omega_-)t)$, \cite{Gaveau1995,Lombardo2014}


\begin{figure}[thb!]
\includegraphics[width=1.0\columnwidth]{P_Delta_0_g_0_2.pdf}
\caption{{\bf Impurity dynamics.} $P_1(t)$ (black, dotted), $P_1^\text{s}(t)$ (red, solid) and $P_1^\text{b}(t)$ (purple, dashed) for $\Delta-\epsilon=0$ and $g=J/5$ in logarithmic scale. In the inset we show $P_1^\text{s}(t)$ in log-log scale with the different contributions: the exponential decay (blue, dashed), the power-law with $t^{-1}$ (green, dotted) and that decaying with $t^{-3}$ (orange, dotted-dashed). We remark that this $t^{-3}$ contribution just makes sense in the infinite time limit, when the other contributions have already decayed. Notice that we remove the oscillations in all the curves for the sake of clarity.}\label{fig:qubit_dynamics}
\end{figure}

We sum up all this information in Fig.\ \ref{fig:qubit_dynamics}, where we plot the impurity dynamics for $\Delta-\epsilon=0$ and $g=J/5$, using logarithmic scale. 
We remove the asymptotic oscillations for the sake of clarity. The population $P_1(t):=|c_1(t)|^2$ is drawn as a black, dotted curve. This curve shows an initial exponential decay which, after a transient period, achieves the stationary regime of $P_1^\text{b}(t)$ (purple, dashed curve). We also show $P_1^\text{s}(t)$ in the red, solid curve. After the initial exponential decay, it decays with a power-law, as expected. We show $P_1^\text{s}(t)$ in log-log scale in the inset, as well as the $t^{-1}$ prediction, which eventually decays with $e^{-t/\tau_1}$ (as $\Delta-\epsilon=0$, $\tau_1:=\tau_{1,+}=\tau_{1,-}$), and the $t^{-3}$ power-law. The agreement with the decay is perfect in this limit, as well as the oscillation with $\cos^2(2t-3\pi/4)$ (not shown).

{\color{red} se puede observar estas tails?} {\color{blue}Quiz\'as Luis sepa decir algo al respecto.}


\section{Conclusions}\label{sec:conclusions}

Summarizing, we have demonstrated that in presence of a band gap and single-photon bound states, spontaneous decay and scattering are no longer symmetric problems, in the sense that the scattering resonance frequency differs from the spontaneous emission frequency. In addition to this, we have seen that the existence of bound states modifies the quantum impurity dynamics in an intriguing way, creating three separate dynamical regimes: an initial exponentially decaying regime, followed by a power-law at long times and eventually ending up in a stationary oscillating regime. Finally, we have also studied how the profile of the emitted photons is affected by the position of the quantum impurity with respect to the band edges.

Some features, such as the spectroscopic shifts in the spontaneously emitted photons can be easily detected by tuning up and down the frequency of the qubit with respect to the band edge. For time-domain experiments and probing the qubit dynamics, we would suggest using a more sophisticated protocol that (i) places the qubit at the right frequency, (ii) then excites it and after a finite time $t$ (iii) detunes the qubit and probes dispersively its excited state population. All of these ideas can be implemented in state-of-the-art setups with superconducting cavities and transmon qubits\ \cite{liu2016}, and also with quantum dots in photonic crystals.


\begin{acknowledgements}
We acknowledge 
support by the Spanish Ministerio de Economia y Competitividad within projects MAT2011-28581-C02, FIS2012-33022 and No. FIS2014-55867-P, the Gobierno
de Aragon (FENOL group), CAM Research Network QUITEMAD+
and the European project PROMISCE.
\end{acknowledgements}

\appendix

\section{Bound States}\label{app:eigen}

%\subsection{Scattering states}

%We start by writing the coefficients of the scattering states (Eq. \ref{eq:scattering_states}). By solving the eigenvalue equation for $|\Psi_k\rangle$ one can obtain that the impurity amplitude $d_k$ is
%\begin{equation}
%d_k=\frac{g t_k}{\omega_k-\Delta}\label{eq:d_scattering_states}.
%\end{equation}
%This is how we obtain the value of $c_k=\langle \Psi_k|\sigma^+|0\rangle=d_k^*$, shown in the main text (Eq. \ref{eq:ck}).

%By imposing continuity of $|\Psi_k\rangle$ at the impurity position, the reflection amplitude is
%\begin{align}
%\label{eq:reflection} r_k & = t_k - 1.
%\end{align}

%\subsection{Bound states}

Let us introduce the expressions of the bound states (Eq. \eqref{eq:bound_states}). $d_\pm$, the impurity amplitude, is
\begin{equation}
d_\pm = \frac{g}{\omega_\pm - \Delta}.\label{eq:d_bound_states}
\end{equation}
The exponential decay rate $\kappa_\pm$ is such that $\text{Re}(\kappa_\pm)>0$ and $\text{Im}(\kappa_\pm)=0,\pi$. Defining $\eta_\pm := e^{-\kappa_\pm}$ and using the eigenvalue equation $H|\Psi_\pm\rangle = \omega_\pm|\Psi_\pm\rangle$
\begin{equation}\label{eq:eta}
\eta_\pm ^4 + \frac{\Delta-\epsilon}{J} \eta_\pm ^3 + \frac{g^2}{J^2} \eta_\pm ^2 - \frac{\Delta-\epsilon}{J} \eta_\pm - 1 = 0.
\end{equation}
This fourth degree algebraic equation has four solutions. However, there are just two solutions for $\eta_\pm$ fulfilling the aforementioned restrictions for $\kappa_\pm$, which eventually define $|\Psi_\pm\rangle$.
The normalization factor is
\begin{equation}\label{eq:Npm}
N_\pm = \left(\frac{1+e^{-2\kappa_\pm}}{1-e^{-2\kappa_\pm}}+|d_\pm|^2\right)^{-1/2}.
\end{equation}
Finally, $c_\pm=\langle 0|\sigma^-|\Psi_\pm\rangle=(N_\pm d_\pm)^*$ can be obtained (Eq. \eqref{eq:cpm}), since we know both $d_\pm$ and $N_\pm$.

%\section{Power expansion of $\delta\omega_\text{ph}$}\label{app:wph}

%First of all, we have to compute $\eta_\pm$ by solving Eq. \ref{eq:eta}. Trivially, we realize that $\eta$ is an even function of $g$. Then, we expand $\eta_\pm$ in $g$:
%\begin{equation}\label{eq:eta_2}
%\eta_\pm\simeq \mp 1 + c_{2,\pm} (g/J)^2 + c_{4,\pm} (g/J)^4.
%\end{equation}
%Introducing this into Eq. \ref{eq:eta} and expanding the resulting equations up to fourth order:
%\begin{align}
%&\left(1-4c_{2,+}+2c_{2,+}\frac{\Delta-\epsilon}{J}\right)\frac{g^2}{J^2}\\
%&+\left(6c_{2,+}^2-2c_{2,+}-4c_{4,+}+(2c_{4,+}-3c_{2,+}^2)\frac{\Delta-\epsilon}{J}\right)\frac{g^4}{J^4} = 0\nonumber,\\
%&\left(1+4c_{2,-}+2c_{2,-}\frac{\Delta-\epsilon}{J}\right)\frac{g^2}{J^2}\\
%&+\left(2c_{2,-}+6c_{2,-}^2+4c_{4,-}+(3c_{2,-}^2 + 2c_{4,-})\frac{\Delta-\epsilon}{J}\right)\frac{g^4}{J^4} = 0\nonumber.
%\end{align}
%We get $c_{n,\pm}$ solving this system:
%\begin{align}
%&c_{2,\pm}=-\frac{J}{2(\Delta - \epsilon \mp 2J)},\\
%&c_{4,\pm}=\mp\frac{J^2}{8(\Delta-\epsilon \mp 2J)^2}.
%\end{align}
%Once we know $\eta_\pm$ (Eq. \ref{eq:eta_2}), we can calculate $\omega_\pm=\epsilon-2J(\eta_\pm^{-1}+\eta_\pm)$ and $|c_\pm|^2$ (Eq. \ref{eq:cpm}) up to fourth order:
%\begin{align}
%&\omega_\pm=\epsilon \pm 2J \pm  \frac{1}{4J(\Delta-\epsilon \mp 2J)^2}g^4,\\
%&|c_\pm|^2=\frac{(g/J)^4}{2J(2J\pm (\Delta-\epsilon))^3}.
%\end{align}
%Introducing this into the expression of $\delta\omega_\text{ph}$ (Eq. \ref{eq:omega_ph}) and expanding again up to fourth order:
%\begin{equation}
%\delta\omega_\text{ph}=\Delta-\frac{4(\Delta-\epsilon)}{((\Delta-\epsilon)^2-4J^2)^2}g^4 + \mathcal{O}\left(g^6/J^5\right),
%\end{equation}
%which is the formula we write in the main text (Eq. \ref{eq:omega_ph_4}).

\section{Emitted field}\label{app:field}

The position profile of the emitted field $\phi_x(t)=\langle 0|a_x|\Psi(t)\rangle$ fulfills the following expression:
\begin{align}
\phi_x(t)& = \frac{1}{2\pi}\int_{-\pi}^\pi dk c_k e^{-i\omega_k t}\langle 0|a_x|\Psi_k\rangle \\
& + c_+ e^{-i\omega_+ t} \langle 0|a_x|\Psi_+\rangle+ c_- e^{-i\omega-+ t} \langle 0|a_x|\Psi_-\rangle, \nonumber
\end{align}
where we have used Eq. \eqref{eq:psi(t)}. In order to compute the amplitude $\langle 0|a_x|\Psi_k\rangle$ we take the expression of $|\Psi_k\rangle$, Eq. \eqref{eq:scattering_states}, for $k>0$:
\begin{equation}
\langle 0|a_x|\Psi_k\rangle = \left\{ 
\begin{array}{c}
e^{ikx}+r_ke^{-ikx}\quad x<0,\\
t_k e^{ikx} \qquad\qquad\;\;\; x\geq 0.
\end{array}
\right.
\end{equation}
If $k<0$:
\begin{equation}
\langle 0|a_x|\Psi_k\rangle = \left\{ 
\begin{array}{c}
t_k e^{ikx} \qquad\qquad\;\;\; x< 0,\\
e^{ikx}+r_ke^{-ikx}\quad x\geq 0.
\end{array}
\right.
\end{equation}
Because of these terms, $|\phi_x(t)|^2$ shows oscillations in position space (Fig. \ref{fig:w_n}).

The amplitudes $\langle 0|a_x|\Psi_\pm\rangle$ are computed by projecting on $|\Psi_\pm\rangle$ (Eq. \eqref{eq:bound_states}):
\begin{equation}
\langle 0|a_x|\Psi_\pm \rangle= N_\pm e^{-\kappa_\pm |x|}.
\end{equation}



\section{Impurity dynamics: analyzing the integrand}\label{app:integrand}

%\begin{align}
%&c_1^\text{s}(t) =  e^{-i\epsilon t}\frac{4g^2}{\pi J^2}\int_{-1}^1 dy F(y)e^{i2yJt}\label{eq:c_sc_app},\\
%&F(y)=\frac{\sqrt{1-y^2}}{4(1-y^2)\left((\Delta-\epsilon)/J+2y\right)^2+(g/J)^4}.\label{eq:F}
%\end{align}
The integrand $F(y)$, Eq. \eqref{eq:F}, depicts a well peaked behavior, with a Lorentzian profile around some value $y$ far from the integration limits when $\Delta$ is far enough from the band edges and the coupling constant is small enough (see red, solid curve of Fig. \ref{fig:integrand}, with $g=J/5$ and $\Delta-\epsilon=0$). In such a case, we can substitute $F(y)$ by $L(y)\sim 1/((y-y_p)(y-y_p^*))$, being $y_p$ the pole corresponding to the peak of $F(y)$, with $-1<\text{Re}(y_p)<1$ and $\text{Im}(y_p)>0$. We extend the integration domain to $\pm\infty$. Then
\begin{equation}
c_1^\text{s}(t) = i8(g/J)^2 e^{-i\epsilon t}e^{i2y_p Jt}\text{Res}(F(y),y=y_p),
\end{equation}
By computing this numerically, we obtain the decay rate  $\gamma_0 = 4J\;\text{Im}(y_p)$ and the oscillating phase $\varphi = \epsilon - 2J\;\text{Re}(y_p)$, as shown in Fig. \ref{fig:qubit_decay} a) and b) respectively in the main text.

%\begin{figure}[thb!]
%\includegraphics[width=1.0\columnwidth]{integrand_inset_Delta=0_g=0,2.pdf}
%\caption{{\bf Integrand.} Integrand $F(y)$ for $g=J/5$ and $\Delta-\epsilon=0$ (red, solid), the same for $\Delta-\epsilon=0$ and $g=0$ (blue, dashed) and $G(y)$ (Eq. \ref{eq:G}, black, dotted). In the inset, we zoom $F(y)$ around $y=-1$ to see $\Delta y_-$.}\label{fig:integrand}
%\end{figure}

However, as $t\gg \tau_0$, different regimes emerge. $F(y)$ has a sharp behavior close to $y=\pm 1$. Actually, it diverges when $y\to\pm 1$ if $g=0$. In order to take into account this contribution, we can approximate $F(y)$ by $F(y)|_{g=0}$ (see blue, dashed curve of Fig. \ref{fig:integrand})
\begin{equation}
c_1^\text{s}(t)\simeq \frac{4g^2}{\pi J^2}\int_{-1}^1 dy \frac{e^{i2yJt}}{4\sqrt{1-y^2}((\Delta-\epsilon)/J+2y)^2}.
\end{equation}
If $2\Delta y_\pm Jt\ll 1$ the oscillatory term $e^{i2yJt}$ will not be sensitive to the difference between $F(y)$ and $F(y)|_{g=0}$ when $y$ is close to the edges and the approximation will work. As we are concerned in the contribution around $\pm 1$, we can approximate the integral as:
\begin{align}
c_1^\text{s}(t)  \simeq \frac{4g^2}{\sqrt{2}\pi J^2}&\left(\frac{J^2}{(\Delta-\epsilon-2J)^2}\int_{-1}^\infty dy \frac{e^{i2yJt}}{4\sqrt{1+y}}\right. \\
& \left.+ \frac{J^2}{(\Delta-\epsilon+2J)^2}\int_{-\infty}^1 dy \frac{e^{i2yJt}}{4\sqrt{1-y}}\right).\nonumber
\end{align}
These integrals are analytical
\begin{equation}
c_1^\text{s}(t)\simeq \frac{g^2}{2\sqrt{2\pi Jt}}\left(\frac{e^{-i2Jt}}{(\Delta-\epsilon-2J)^2}+\frac{e^{i2Jt}}{(\Delta-\epsilon+2J)^2}\right).\label{eq:csc_1}
\end{equation}
Then, $P_1^\text{s}(t)$ decays with $(Jt)^{-1}$ after the initial exponential decay if $\tau_0\ll t\ll \tau_{1,\pm}$, with $\tau_{1,\pm}:=(4|\Delta y_\pm|)^{-1}$. We can rewrite the last expression by adding the decaying exponentials with $\tau_{1,\pm}$:
\begin{equation}
c_1^\text{s}(t)\simeq \frac{g^2}{2\sqrt{2\pi Jt}}\left(\frac{e^{-i2Jt}e^{-\frac{t}{2\tau_{1,-}}}}{(\Delta-\epsilon-2J)^2}+\frac{e^{i2Jt}e^{-\frac{t}{2\tau_{1,+}}}}{(\Delta-\epsilon+2J)^2}\right).
\end{equation}
In the case of the Fig. \ref{fig:qubit_dynamics}, $\Delta-\epsilon=0$, thus $\tau_1:=\tau_{1,+}=\tau_{1,-}$.

Finally, if $t\gg \tau_0,\tau_{1,\pm}$, the only surviving contributions will be due to the singularities of $F(y)$ because the oscillating term $e^{i2yJt}$ smears out any soft part of the kernel. $F(y)$ is singular at $y=\pm 1$. Then, we can approximate the kernel by any function which behaves as $F(y)$ when $y=\pm 1$. We consider the function $G(y)$ (see Fig. \ref{fig:integrand}, black, dotted curve)
\begin{equation}
G(y)=\frac{\sqrt{1-y^2}}{(g/J)^4}.\label{eq:G}
\end{equation}
It is the same function as $F(y)$ neglecting the $y$-dependent part of the denominator. Then, if $t\gg \tau_0,\tau_{1,\pm}$
\begin{equation}
c_1^\text{s}(t) \simeq  \frac{4J^2}{\pi g^2}\int_{-1}^1 dy \sqrt{1-y^2} e^{i2yJt}=\frac{2J}{g^2 }\frac{J_1(2Jt)}{t},
\label{eq:c_sc_bessel}
\end{equation}
being $J_1$ the first kind Bessel function with $n=1$. As $t\to\infty$, $J_1(2Jt)\to (\pi Jt)^{-1/2}\cos(2Jt-3\pi /4)$, so $P_1^\text{s}(t)$ decays with $(Jt)^{-3}$ in the long-time limit.

%If we consider $|n|\gg 1/\text{Re}(\kappa_\pm)$ and $n>0$:
%\begin{align}
%\phi_n(t)& \simeq \frac{1}{\sqrt{2\pi}}\int_{-\pi}^0 dk c_k e^{-i\omega_k t}(e^{ikn} + r_ke^{-ikn}) \nonumber\\
%& + \frac{1}{\sqrt{2\pi}}\int_0^\pi dk c_k e^{-i\omega_k t}t_ke^{ikn}.
%\end{align}

%\section{Non-Markovianity}
%
%As said in the main text, the qubit dynamics is not the typical exponential decay. In this appendix, we argument why this dynamics is, in addition, non-Markovian.\newline
%
%\begin{figure}[thb!]
%\includegraphics[width=1.0\columnwidth]{Gamma_qb_Delta=0,80_g=0,30.pdf}
%\caption{{\bf Qubit dynamics.} $\Gamma$ for $\Delta=0.80$ and $g=0.30$ in logarithmic scale.}\label{fig:Gamma}
%\end{figure}
%
%It is possible to modelize the qubit dynamics as a Lindblad one without Hamiltonian term, leading to
%\begin{equation}
%\frac{d|c(t)|^2}{dt} = -\Gamma(t) |c(t)|^2,
%\end{equation}
%being $\Gamma(t)$ the decay rate. If $\Gamma(t)=\Gamma_0$ for all $t$, we have the typical exponential behavior. However, this is not our case. As introduced by Rivas et al in \cite{Plenio2010}, we talk about Non-Markovian dynamics if $\Gamma(t)<0$. Since our dynamics has revivals, we can anticpate that it will be non-Markovian. We show $\Gamma(t)$ for $\Delta=0.80$ and $g=0.30$ (same parameters as in Fig. \ref{fig:qubit_dynamics}) in Fig. \ref{fig:Gamma}. As seen, in the beginning, corresponding to the initial decay, $\Gamma(t)>0$. However, $\Gamma(t)$ clearly takes negative values, following oscillations in the stationary regime, in the same way as $P_\text{qb}(t)$, as expected. So, we conclude that this system has a Non-Markovian behavior.


\bibliographystyle{apsrev4-1}
\bibliography{scattering_eduardo} 

\end{document}
